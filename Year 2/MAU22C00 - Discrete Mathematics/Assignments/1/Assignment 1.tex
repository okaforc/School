\input{settings} % add packages, settings, and declarations in settings.tex
\DeclareUnicodeCharacter{2212}{-}
\begin{document}

\lhead{Chike Okafor}
\rhead{MAU22C00 Assignment 1} 
\cfoot{\thepage\ of \pageref{LastPage}}

 \begin{enumerate}
   \item Prove via inclusion in both directions that for any three 
   sets A, B, and C 
    \[(A \cup B) \times C = (A \times C) \cup (B \times C)\]
    
    \textbf{Solution: }
    First, we must show that \[(A \cup B) \times C \subseteq (A \times C) \cup (B \times C).\] We know that sets have the distributive property due to Tautology \#29, which states that \[P \lor (Q \land R) \iff [(P \lor Q) \land (P \lor R)]\]
    we know know that  \[C \lor (A \land B) \iff [(C \lor A) \land (C \lor B)].\]
    So \[(A \cup B) \times C = (A \times C) \cup (B \times C)\] is proven to be true. Next we must show that \[(A \times C) \cup (B \times C) \subseteq (A \cup B) \times C. \] Since Tautology \#29 works in reverse as well, this too holds.
    
    Since \((A \times C) \cup (B \times C) \subseteq (A \cup B) \times C \) is true and \((A \cup B) \times C \subseteq (A \times C) \cup (B \times C)\) is true, we have shown that \((A \cup B) \times C = (A \times C) \cup (B \times C)\) is true. 
    \\\\
    \item Let A be the set of all people who have ever lived. For \(x, y, \ \in  A, \ xRy\) if and only if \(x\) and \(y\) share at least one parent. Determine 
    \begin{enumerate}
        \item Whether or not the relation \(R\) is reflexive;
        \item Whether or not the relation \(R\) is symmetric;
        \item Whether or not the relation \(R\) is anti-symmetric;
        \item Whether or not the relation \(R\) is transitive;
        \item Whether or not the relation \(R\) is an equivalence relation;
        \item Whether or not the relation \(R\) is a partial order.
        \\\\
    \end{enumerate}
    
    \textbf{Solution: }
    \begin{enumerate}
        \item Yes, \(R\) is reflexive. \(\forall x\), \(x\) shares their parents with themselves. \(x \cap x = x.\)
        \item \(R\) is symmetric. \(\forall x, y \ \in \ A\), if \(x\) shares one parent with \(y\), then \(y\) must share a parent with \(x\). \(x \cap y = y \cap x.\)
        \item \(R\) is not anti-symmetric. R is anti-symmetric iff \(xRy \land yRx \implies x = y.\) In other words, R is anti-symmetric only if one person can have a certain set of parents. We know this is not true as we have already proven A is symmetric, showing that \(x\) and \(y\) do not need to be equal to share a parent. 
        \item \(R\) is not transitive. Assume \(x\) has parents \(i\) and \(j\), \(y\) has parents \(j\) and \(k\), and \(z\) has parents \(k\) and \(l\). For transitivity to hold, \(z\) must either have \(i\) or \(j\) as a parent. In this case, xRy and yRz hold, but xRz does not, so R is not transitive.
        \item \(R\) is not an equivalence relation. To be an equivalence relation, R must be symmetric, reflexive, and transitive. As \(R\) is not transitive, it is not an equivalence relation.
        \item \(R\) is not a partial order. To be a partial order, \(R\) must be reflexive, transitive, and anti-symmetric. As \(R\) is not anti-symmetric or transitive, it is not a partial order.
        \\\\
    \end{enumerate}
    
    \item Let \(f : [−1, 1]\) \(\mapsto [−1, 0]\) be the function defined by
    \(f(x) = x^2 - 1\) for all \(x \in [−1, 1].\) Determine whether or not this function is injective and whether or not it is surjective.
    \\\\
    \textbf{Solution: } \\
    End points: 
    \[x = -1: (-1)^2 - 1= 1 - 1 = 0\]
    \[x = 1: (1)^2 - 1= 1 - 1 = 0\]
    \underline{Injective} \\
    \[f'(x^2 - 1) = x = 0\] %set x to zero for derivatives
    \[f''(x^2 - 1) = f'(x) = 1 > 0\] 
    So there is a local minimum at \(x = 0\). Substituting \(x\) into \(f(x)\) gets us:
    \[f(0) = (0)^2 - 1 = -1.\] 
    So \(\exists x \in [0, 1]\) s.t. \(f(x) = -1,\) as \(-1 \in [-1, 0] = [f(0, 1)].\) Let \(x^2 - 1 = 0.\) Then
    \[(x+1)(x-1) = 0.\] Therefore \(f(1) = f(-1).\) Since \(1 \ne -1\), \( f(x)\) is not injective.
    \\\\
    \underline{Surjective} \\
    The local minimum of \(f(x)\) is \(-1\) at \(x = 0\). The values at the end points were also found to be \(f(-1) = 0\) and \(f(1) = 0\). Therefore \(-1\) is the global minimum. Let \(f(x) = y.\) Then 
    \[y = x^2 - 1\]
    \[y + 1 = x^2\]
    \[\sqrt{y + 1} = x\]
    Then \(f(x) = f(\sqrt{y + 1}) = (\sqrt{y + 1}^2 - 1) = y.\) Sqauring a square root removes the square, so we are left with \(y + 1 - 1 = y.\) Since we are left with \(y = y,\) we know that \(f\) is surjective.
    \\\\
    \item Prove by mathematical induction that if \(k \in \mathbb{N} \) and \(k > 2,\) then \(2^k > 1 + 2k.\)\\
    \textbf{Solution:} Fix \(k \in N\)\\
    \textbf{Base case:} \(k = 3.\)\\
    Then 
    \[2^3 > 1 + 2(3) = 8 > 7\]
    as required.\\
    \textbf{Induction step: Assume true for} \(n = k.\)\\
    \textbf{Prove true for} \(n = k + 1.\)\\
    \[2^k \cdot 2 > 2(2k+1) = 4k+2\]
    \[4k = 2k + 2k > 2k + 1\]
    \[= 4k+2 > 2k+3\]
    \[= 2^{k+1} > 4k+2 > 2k+3\]
    \[= 2^{k+1} >2k+3\]
    as required.
    \\\\
    \item Let \(A =\{z \in \mathbb{C} \ | \ z^6 = 1\}\) with the operation of multiplication. 
    \begin{enumerate}
        \item Is \((A, \cdot)\) a semigroup?
        \item Is \((A, \cdot)\) a monoid?
        \item Is \((A, \cdot)\) a group?
        \item Write down an isomorphism between \((A, \cdot)\) and \((\mathbb{Z}_{6}, \oplus_6)\).
        \\\\
    \end{enumerate}
    
    \textbf{Solution:}
    \begin{enumerate}
        \item Yes, \((A, \cdot)\) is a semigroup. In order to be a semigroup, \(A\) must be endowed with an associative binary operation. To prove \(\cdot\) is associative, let \(x=a+bi, y=c+di,\) and \(z=e+fi,\) where \(x,y,z \in \mathbb{C}\) and \(x^6 = y^6 = z^6 = 1.\) If \(\cdot\) is associative, then \(x(yz) = (xy)z.\) In other words, 
        \[(a+bi)[(c+di)(e+fi)] = [(a+bi)(c+di)](e+fi).\]
        So
        \[(a+bi)[(c+di)(e+fi)\]
        \[=(a+bi)[c(e+fi) + di(e+fi)]\]
        \[=(a+bi)[ce + cfi + dei - df]\]
        \[=(a+bi)(ce - df + (cf + de)i)\]
        \[=a(ce-df+(cf+de)i) + bi(ce-df+(cf+de)i)\]
        \[=ace-adf+acfi+adei+bcei-bdfi-bcf-bde\]
        \[=ace-adf+bcf+bde+(acf+ade+bce-bdf)i\]
        \[=[e(ac-bd) + f(ad-bc)] + [e(ad+bc) + f(ac-bd)]i\]
        \[=(e+fi)[(ac-db)+(ad+bc)i]\]
        \[=(e+fi)[(a+bi)(c+di)]\]
        \[=[(a+bi)(c+di)](e+fi).\]
        
        Thus \((a+bi)[(c+di)(e+fi)] = [(a+bi)(c+di)](e+fi)\) as required.
        
        \item Yes, \((A, \cdot)\) is a monoid. The identity element \(e\) under multiplication is \(1\). \\
        Proof: 
        \[1 = 1 + 0i\]
        \[(a+bi)(1+0i) = a+bi\]
        \[=a(1+0i) + bi(1+0i)\]
        \[=a(1) + bi(1)\]
        \[=a + bi\]
        Since \(1+0i \in \mathbb{C}\) and \((1)^6 = 1, A\) is a monoid.
        
        \item If \(A\) is a group, then it must be a monoid and every element in \(A\) must be invertible. Let \(z \in \mathbb{C},\) where \(z^6 = 1.\) Let \(z^{-1}\) be the inverse of \(z\), such that \(zz^{-1} = z^{-1}z = 1.\) \(z\) can be written in the form \(a+bi\), where \(a,b \in \mathbb{R}\). So
        \[z^{-1}(a+bi) = 1\]
        \[z^{-1} = \frac{1}{a+bi}\]
        \[=\frac{a-bi}{(a+bi)(a-bi)}\]
        \[=\frac{a-bi}{a^2 + b^2}\]
        So \(z^{-1} = \frac{a-bi}{a^2+b^2}\). To confirm this, we will test if \(z^{-1}z = 1\).
        \[zz^{-1} = (a+bi)\frac{a-bi}{a^2+b^2}\]
        \[=\frac{a^2 - abi + abi - b^2i^2}{a^2 + b^2}\]
        \[=\frac{a^2+b^2}{a^2+b^2}\]
        \[=1\]
        So as long as \(a^2 + b^2 \ne 0\), there exists an inverse of \(z \in \mathbb{C}\). Since \((0)^6 = 0 \ne 1\) and \( 0 \not\in A,\) it is a group.
        
        \item An isomorphism between \((A, \cdot)\) and \((\mathbb{Z}_{6}, \oplus_6)\) is 
        \[f(k) = cos\left(\frac{2\pi k}{6}\right) + i\cdot sin\left(\frac{2\pi k}{6}\right)\]
        To justify this, take some \(z \in \mathbb{C}\) such that \(z^6=1.\) According to De Moivre's theorem, \(z^k = r^k(\cos{k\theta} + i\cdot \sin{k\theta}) = r^ke^{ki\theta}\), so then \(e^{ki\theta} = \cos{k\theta} + i\cdot\sin{k\theta}\), for some \(k \in \mathbb{Z}\). Let \(\theta=2\pi.\) Then 
        \[e^{2\pi ik}=\cos{2\pi ik} + i\cdot\sin{2\pi ik}\]
        \[=1\]
        Then \(e^{2\pi ik}=1.\) According to De Moivre's theorem, \(z^6 = r^6e^{6i\theta}=1\). For any \(z,\ z=a+bi,\ a,b \in \mathbb{R}.\) Since \(r =|\sqrt{a^2 + b^2}|\) is a positive real number and for any \(z^n,\ n \in \mathbb{Z},\ z^n=1,\ r^n = 1.\) So \(r = 1\) and \(e^{6i\theta} = e^{2\pi ik}\). Taking the natural logarithm of of both sides gets us \(6i\theta = 2\pi ik\). Solving for \(\theta\), we end up with 
        \[\theta = \frac{2\pi k}{6}\]
        Substituting this back into trigonometric form gets us 
        \[z = cos\left(\frac{2\pi k}{6}\right) + i\cdot sin\left(\frac{2\pi k}{6}\right)\]
        for some integer \(k\). So any \(z \in \mathbb{C}\) where \(z^6=1\) can be expressed as this formula given some integer \(k\). Substituting \(\{0, 1, 2, \ldots, 5\}\) into \(k\) returns each unique root of \(z\). If \(k > 5\), the results repeat. In other words, for some integer \(k = \{0, 1, 2, \ldots, 5\}\) using a number greater than \(n-1\) still returns a root of \(z\). For example, \(z\) when \(k=3\) is the same as \(z\) when \(k=9\), or \(3 \equiv 9 \pmod{6}\). So 
        \[f(k) = cos\left(\frac{2\pi k}{6}\right) + i\cdot sin\left(\frac{2\pi k}{6}\right)\]
        is an isomorphism from \((\mathbb{Z}_{6}, \oplus_6)\) to \((A, \cdot)\) as any \(f(a) \cdot f(b) = f(a \oplus b)\) and each \(k\) gives a unique root of \(z\).
        
    \end{enumerate}
 \end{enumerate}

\end{document}
